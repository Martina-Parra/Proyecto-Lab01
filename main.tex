\documentclass{article}
\usepackage{amssymb}
\usepackage[table,xcdraw]{xcolor}
\usepackage{amsmath}
\usepackage[utf8]{inputenc}
\usepackage[spanish]{babel}
\usepackage[sc]{mathpazo} 
\linespread{1.05}
\usepackage{microtype}
\usepackage[hang, small,labelfont=bf,up,textfont=it,up]{caption}
\usepackage{lettrine}
\usepackage{graphicx}
\usepackage[hmarginratio=1:1,top=20mm,bottom=35mm,right=15mm,left=15mm,columnsep=20pt]{geometry}
\usepackage{multicol} 
\usepackage{booktabs}
\usepackage{float} 
\usepackage{subfigure}
\usepackage{paralist} 
\usepackage{hyperref} 
\usepackage{abstract}
\renewcommand{\abstractnamefont}{\normalfont\bfseries}
\renewcommand{\abstracttextfont}{\normalfont\small\itshape}
\usepackage{titlesec}
\renewcommand\thesection{\Roman{section}} 
\renewcommand\thesubsection{\Roman{subsection}} 
\titleformat{\section}[block]{\large\scshape\centering}{\thesection.}{1em}{} 
\titleformat{\subsection}[block]{\large}{\thesubsection.}{1em}{}

\usepackage{fancyhdr} 



\fancyfoot[RO,LE]{\thepage} \newenvironment{Figure}
  {\par\medskip\noindent\minipage{\linewidth}}
  {\endminipage\par\medskip}



\title{\vspace{-18mm}\fontsize{18pt}{20pt}\selectfont\textbf{Intensidad del campo magnético de un imán de neodímio a distintas temperaturas}} 


\author{
\large
\textsc{Noemí de la peña, Benjamín Opazo, Martina Contreras, Jeremías Martínez}\\[2mm]
\large Estudiantes de Cs. Físicas, Universidad de Concepción. 
%\vspace{2mm}
}
\date{}




\begin{document}

\pagestyle{fancy}
\lhead[]{Laboratorio I}
\chead[]{}
\rhead[]{Experimento}

\maketitle
\headrule

\begin{abstract}
    En el siguiente informe, se muestran los resultados del estúdio del campo magnético de un imán de neodimio y su reacción al modificar la temperatura de este metal. Mediante un análisis cuantitativo de las variables de interés examinaremos la situación mediante la recolección empírica de datos del campo obtenidos por medición directa utilizando el magnétometro incluido como herramienta de phyphox\cite{ref2}, al estudiar una porción pequeña de dicho material a distintas temperaturas se puede inferir de los datos registrados, que a medida que la temperatura incrementa, el campo magnético disminuye, y de forma análoga al disminuir la temperatura en el imán, su intensidad magnética aumenta.
\end{abstract}



\begin{multicols}{2}
\section{Introducción}
El imán y sus propiedades magnéticas fueron descubiertas por Tales de mineto, en los años 600 A.C.
El descubrimiento de imán dividido en un antes y después la historia de la humanidad, dado como origen a 2 ramas físicas muy importantes en la actualidad , como: El magnetismo y el electromagnetismo.

En este informe presentaremos una variedad de datos , relacionados con la variación del campo magnético a distintas temperaturas.
Donde primero presentaremos los objetivos de nuestra investigación. Luego definir algunos conceptos importantes, para entender con claridad lo sucedido en el laboratorio. Después expondremos los materiales y procedimientos utilizados. 
Finalizaremos exponiendo los datos, tanto en tablas como en graficas. Y en consiguiente, diremos si se cumplieron los objetivos de nuestra investigación.
\section{Marco teórico}


\section{Procedimiento}
\begin{enumerate}
    \item Primero,llenamos el vaso precipitado con 400 ml de agua, luego colocamos el imán en su interior.
    \item Ahora con ayuda del mechero calentamos el envase precipitado junto con el  imán, hasta alcanzar una temperatura de 10°C
    \item Con guantes y mucho cuidado alejamos el vaso precipitado del mechero, posteriormente esperamos 2 minutos para que el imán y el agua alcancen un equilibrio térmico.
    \item Con ayuda de unas pinzas, sacamos el imán del interior del vaso precipitado y calculamos su campo magnético utilizando la aplicación pyphox, a una distancia de aproximadamente 3 [cm] del celular. (repetimos pasos 2,3,4 para temperatura 20°,30°,40°,50°,60°,70°,80°,90° y 100°)
    \item De manera paralela, colocamos un imán idéntico al anterior, en el interior de un recipiente lleno de agua y lo metemos adentro de una nevera, hasta que alcance una temperatura cercana a 0°.(repetimos para una temperatura cercana a 10°)
    \item Sacamos el imán  del recipiente ,enfriado y calculamos su campo magnetico con pyphox.
\end{enumerate}
\section{Materiales}
\begin{itemize}
    \item Imán de neodimio (X2)
    \item Vaso precipitado (1000 [ml])
    \item Agua
    \item Termómetro (Termocupla)
    \item Mechero
    %\item Soporte universal
    \item Pinzas
    \item Regla metálica
    \item Magnetómetro (En Phyphox)
\end{itemize}

\section{Análisis}
\end{multicols}


\begin{thebibliography}{}


\bibitem[1]{ref1} Alcázar, G. A. P. (2016, 3 julio). \textit{ Imanes permanentes: características, aplicaciones y futuro | Revista de la Academia Colombiana de Ciencias Exactas, Físicas y Naturales}.  from \\
\hyperlink{https://raccefyn.co/index.php/raccefyn/article/view/361}{https://raccefyn.co/index.php/raccefyn/article/view/361}

\bibitem[2]{ref2} Staacks, S. (n.d.). Your smartphone is a mobile lab. Phyphox. Retrieved October 27, 2022, from \\
\hyperlink{https://phyphox.org}{https://phyphox.org}

\bibitem[3]{ref3} Kirchmayr, H. R. (1966). Permanent magnets and hard magnetic materials. Journal of Physics D: Applied Physics. 
\hyperlink{https://iopscience.iop.org/article/10.1088/0022-3727/29/11/007}{https://iopscience.iop.org/article/10.1088/0022-3727/29/11/007}










\end{thebibliography}

\end{document}
