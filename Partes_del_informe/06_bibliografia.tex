\begin{thebibliography}{}


\bibitem[1]{ref1} Alcázar, G. A. P. (2016, 3 julio). \textit{ Imanes permanentes: características, aplicaciones y futuro | Revista de la Academia Colombiana de Ciencias Exactas, Físicas y Naturales}.  from \\
\hyperlink{https://raccefyn.co/index.php/raccefyn/article/view/361}{https://raccefyn.co/index.php/raccefyn/article/view/361}

\bibitem[2]{ref2} Staacks, S. (n.d.). Your smartphone is a mobile lab. Phyphox. Retrieved October 27, 2022, from \\
\hyperlink{https://phyphox.org}{https://phyphox.org}

\bibitem[3]{ref3} Kirchmayr, H. R. (1966). Permanent magnets and hard magnetic materials. Journal of Physics D: Applied Physics. 
\hyperlink{https://iopscience.iop.org/article/10.1088/0022-3727/29/11/007}{https://iopscience.iop.org/article/10.1088/0022-3727/29/11/007}










\end{thebibliography}