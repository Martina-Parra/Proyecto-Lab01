\section{Procedimiento}
\begin{enumerate}
    \item Primero,llenamos el vaso precipitado con 400 ml de agua, luego colocamos el imán en su interior.
    \item Ahora con ayuda del mechero calentamos el envase precipitado junto con el  imán, hasta alcanzar una temperatura de 10°C
    \item Con guantes y mucho cuidado alejamos el vaso precipitado del mechero, posteriormente esperamos 2 minutos para que el imán y el agua alcancen un equilibrio térmico.
    \item Con ayuda de unas pinzas, sacamos el imán del interior del vaso precipitado y calculamos su campo magnético utilizando la aplicación pyphox, a una distancia de aproximadamente 3 [cm] del celular. (repetimos pasos 2,3,4 para temperatura 20°,30°,40°,50°,60°,70°,80°,90° y 100°)
    \item De manera paralela, colocamos un imán idéntico al anterior, en el interior de un recipiente lleno de agua y lo metemos adentro de una nevera, hasta que alcance una temperatura cercana a 0°.(repetimos para una temperatura cercana a 10°)
    \item Sacamos el imán  del recipiente ,enfriado y calculamos su campo magnetico con pyphox.
\end{enumerate}