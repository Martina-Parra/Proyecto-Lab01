\section{Procedimiento}
\subsection{Montaje experimental}



%\begin{itemize}
%\item{Imán 1}
%(incluir imágenes)

%\item{Imán 2}
%(incluir imágenes)
%\end{itemize}



\begin{enumerate}
\subsection{Pasos a seguir}
    \item Primero, llenamos el vaso precipitado con $400 ml$ de agua, luego colocamos el imán en su interior.
    \item Ahora con ayuda del mechero calentamos el envase precipitado junto con el  imán, hasta alcanzar una temperatura de $10^\circ C$
    \item Con guantes y mucho cuidado alejamos el vaso precipitado del mechero, posteriormente esperamos 2 minutos para que el imán y el agua alcancen un equilibrio térmico. 
    
    \item Con ayuda de unas pinzas, sacamos el imán del interior del vaso precipitado y calculamos su campo magnético utilizando la aplicación pyphox, a una distancia de aproximadamente 3 [cm] del celular. (repetimos los pasos 2, 3, 4 para las temperaturas $20 [^\circ C]$, $30 [^\circ C]$, $40 [^\circ C]$, $50 [^\circ C]$, $60 [^\circ C]$, $70^\circ$, $80 [^\circ C]$, $90 [^\circ C]$ y $100 [^\circ C]$)
    \item De manera paralela, colocamos un imán idéntico al anterior, en el interior de un recipiente lleno de agua y lo metemos dentro de una nevera, hasta que alcance una temperatura cercana a $0 [^\circ C]$.(repetimos para una temperatura cercana a $10 [^\circ C]$)
    \item Sacamos el imán  del recipiente ,enfriado y calculamos su campo magnético con pyphox.


\end{enumerate}

\begin{figure}[h!]
    \centering
    \includegraphics[width=5cm]{imagenes/Forma de medir.pdf}
    \caption{Método de medición del campo}
    \label{Metodo-de-medicion}
\end{figure}


