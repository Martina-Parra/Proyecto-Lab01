\begin{abstract}
    En el siguiente informe, se muestran los resultados del estudio del campo magnético de un imán de neodimio y su reacción al modificar la temperatura de este metal. Mediante un análisis cuantitativo de las variables de interés examinaremos la situación a través de la recolección de datos utilizando el magnetómetro incluido como herramienta de phyphox \cite{ref2}. Al estudiar una porción pequeña de dicho material a distintas temperaturas se puede inferir de los datos registrados, que a medida que la temperatura incrementa, el campo magnético disminuye, y de forma análoga al disminuir la temperatura en el imán, su intensidad magnética aumenta.

    %del campo obtenidos por medición directa 
\end{abstract}