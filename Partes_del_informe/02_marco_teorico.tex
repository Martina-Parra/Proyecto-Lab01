\section{Marco teórico}
\begin{itemize}
    \item \textbf{Momento magnético} es la fuerza magnética y la orientación de un imán u otro objeto que produce un campo magnético. Los ejemplos de objetos que tienen momentos magnéticos incluyen bucles de corriente eléctrica, imanes permanentes, partículas elementales, varias moléculas y muchos objetos astronómicos.
    \item \textbf{Temperatura de Curie} es la temperatura por encima de la cual un cuerpo ferromagnético pierde su magnetismo, comportándose como un material puramente paramagnético. Esta temperatura característica lleva el nombre del físico francés Pierre Curie, que la descubrió en 1895.
    \item \textbf{Spin}: Propiedad física de las partículas elementales por el cual tienen un momento angular intrínseco de valor fijo. Las partículas con espín presentan un momento magnético, recordando a un cuerpo cargado eléctricamente en rotación, en general el ferromagnetismo que poseen algunos elementos, surge del alineamiento de los espines.
\end{itemize}
